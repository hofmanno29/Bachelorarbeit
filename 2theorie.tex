\chapter{Theoretische Grunlagen}
In diesem Kapitel werden die theoretischen Grundlagen, die zum Verständnis dieser Arbeit essentiell sind, weiter erläutert. Hierbei wird vor allem auf die Aspekte der Virtualisierung und der Container eingegangen.
\section{Virtualisierung}
Die uns heute bekannte Form der Virtualisierung in der Computertechnik hat ihre frühen Wurzeln bereits in den 50er-Jahren. Christopher Strachey beschreibt in seiner Abhandlung \textit{Time Sharing in Large Fast Computers} von 1959 erstmals eine Methode bei der Rechenzeit von Großcomputern sinnvoll und kosteneffizient auf verschiedene Nutzer aufgeteilt wird.\autocite[Vgl.][]{McCarthy.1983} Da zu dieser Zeit Computer sehr kostenintensiv waren, nutzten viele Unternehmen eher die Möglichkeit Rechenzeit an Großcomputern zu mieten, statt sich selbst einen Computer zu kaufen. Dieser Gedanke der Ressourcenteilung, um eine maximale Effizienz von Computersystemen zu erzielen, ist auch heute noch einer der Hauptgründe für Virtualisierung von Maschinen.\\
\\
Der Begriff der Virtualisierung in der Computertechnik hat seine frühen Wurzeln bereits in den 50er-Jahren. Christopher Strachey beschreibt in seiner Abhandlung \textit{Time Sharing in Large Fast Computers} von 1959 erstmals eine Methode bei der Rechenzeit von Großcomputern sinnvoll und kosteneffizient auf verschiedene Nutzer aufgeteilt wird. Da zu dieser Zeit Computer sehr kostenintensiv waren, nutzten viele Unternehmen eher die Möglichkeit Rechenzeit an Großcomputern zu mieten, statt sich selbst einen Computer zu kaufen.\autocite[Vgl.][S. 18ff.]{Docker.2016}Diese Resourcenteilung wird heutzutage als die erste Form der uns bekannten Virtualisierung angesehen.\\
1963 hat das Unternehmen \textit{\ac{ibm}}

\section{Container}
